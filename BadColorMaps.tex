\documentclass[letterpaper,twocolumn,fleqn]{article} 

\pagestyle{empty}                % no page numbers is default

\usepackage{amsfonts}
\usepackage{amssymb}
\usepackage[cmex10]{amsmath}
\usepackage{booktabs}
\usepackage{caption}
\usepackage{enumitem}
\usepackage{graphicx}
\usepackage{fancyvrb}
\usepackage{framed}
\usepackage{ifthen}
\usepackage{cite}
\usepackage{tabulary}
\usepackage{url}
\usepackage{xspace}
\usepackage{wrapfig}
\usepackage[pdfborder={0 0 0}]{hyperref}
\usepackage{verbatim}

\usepackage{ist} % Style for IST in package instead of document class.

\usepackage{color}
\definecolor{yellow}{rgb}{1,1,0}
\definecolor{black}{rgb}{0,0,0}
\definecolor{ltcyan}{rgb}{.75,1,1}
\definecolor{red}{rgb}{1,0,0}
\definecolor{gray}{rgb}{.6,.6,.6}
\definecolor{darkred}{rgb}{0.5,0,0}
\definecolor{darkgreen}{rgb}{0,0.5,0}

% Cite commands I use to abstract away the different ways to reference an
% entry in the bibliography (superscripts, numbers, dates, or author
% abbreviations).  \scite is a short cite that is used immediately after
% when the authors are mentioned.  \lcite is a full citation that is used
% anywhere.  Both should be used right next to the text being cited without
% any spacing. \hcite is a citation that I am hiding, perhaps because I am
% nearing the maximum number of citations for a journal.
\newcommand*{\lcite}[1]{~\cite{#1}}
\newcommand*{\scite}[1]{~\cite{#1}}
\newcommand*{\hcite}[1]{}

\newcommand{\etal}{et al.\xspace}

\newcommand*{\keyterm}[1]{\emph{#1}}

\newcommand{\fix}[1]{{\color{red}\textsc{[#1]}}}
%\newcommand{\fix}[1]{}

% Avoid putting figures on their own page.
\renewcommand{\textfraction}{0.05}
\renewcommand{\topfraction}{0.95}
\renewcommand{\bottomfraction}{0.95}

% Make sure this is big enough so that only big figures end up on their own
% page but small enough so that if a figure does have to be on its own
% page, it won't push everything to the bottom because it's not big enough
% to have its own page.
\renewcommand{\floatpagefraction}{.75}


%%%%%%%%%%%%%%%%%%%%%%%%%%%%%%%%%%
% Title and Authors
%%%%%%%%%%%%%%%%%%%%%%%%%%%%%%%%%%

\title{Why We Use Bad Color Maps and What You Can Do About It}

\author{Kenneth Moreland; Sandia National Laboratories; Albuquerque, New
  Mexico, USA}

\date{} % date has an empty field.

% correct for bad hyphenation here
\hyphenation{}

%%%%%%%%%%%%%%%%%%%%%%%%%%%%%%%%%%
% Begin document
%%%%%%%%%%%%%%%%%%%%%%%%%%%%%%%%%%
\begin{document} 

\maketitle 

\thispagestyle{empty} % prevents the first page to be numbered

%%%%%%%%%%%%%%%%%%%%%%%%%%%%%%%%%%
% Abstract
%%%%%%%%%%%%%%%%%%%%%%%%%%%%%%%%%%

\begin{abstract}
Please include a brief abstract of this paper. Avoid using figures or
equations in the abstract.
\end{abstract}


%%%%%%%%%%%%%%%%%%%%%%%%%%%%%%%%%%%%
% Overall Document Guidelines: Head
%%%%%%%%%%%%%%%%%%%%%%%%%%%%%%%%%%%%
\section{Overall Document Guidelines: Head}
\label{sec:intro}

Clear document of any fonts other than Times, Arial, and Symbol. The
paper should be formatted using the tags provided in the template,
i.e., title, author, section, subsection, subsubsection, eq./fig.,
references, etc.  in a 2 column format on US letter size paper ($8.5
\times 11$ inches, or $21.6 \times 27.9$ cm).

The left and right margins are set automatically to .75 inch (1.90 cm),
and the top and bottom margins to 1.0 inch (2.54 cm). The document is in
a 2-column format with column widths set at 3.38 inch (8.57 cm) and the
gutter -the space between columns- at .25 inch (0.635 cm).

Papers should be a maximum of 4-6 pages; longer papers will be
returned for revision. Please do not place folios or page numbers in
your paper. That information is inserted when we assemble the book.


%%%%%%%%%%%%%%%%%%%%%%%%%%%%%%%%%%
% Graphics and Equations
%%%%%%%%%%%%%%%%%%%%%%%%%%%%%%%%%%
\section{Graphics and Equations}
Graphics and equations should fit within one column (3.38 inches
wide), but full width (7 inch) figures are also acceptable. Equations,
figures and figure captions each have their own style tags. Equations
are numbered using parentheses flushed right as shown below.

\begin{equation}
\label{eq:ist}
\textrm{IS\&T} + \textrm{members} \times \textrm{Confs.} = \textrm{Success}
\end{equation}


\subsection{Helpful Hints and Style Tags: Subhead}
For a complete listing of the style tags for use in this template
refer to Table \ref{tab:fonts}. These are the style tags for
conference proceedings; if you use the wrong template/style tags your
paper will be sent back to you to be reformatted.  All of these forms
and templates related to the publication of conference papers are
available at

www.imaging.org/conferences/guidelines.cfm. 

Select the specific conference and download the Authors Kit. The
template may vary from one conference to another.

The template is set up for MS Word and LaTeX. Please check the paper
carefully to confirm that the styles have been applied correctly, then
print it out and double check to ensure that the paper appears as
intended.

\begin{table}[!h]
\caption{Style Tag Table: Table head}
\label{tab:fonts}
\begin{center}       
\begin{tabular}{|p{0.45\columnwidth}|p{0.45\columnwidth}|} 
\hline
1. Title& title environment \\ \hline
2. Byline: & \\
Author/Affiliation & author environment \\ \hline
3. Head & section environment \\ \hline
4. Subhead & subsection environment \\ \hline
5. Tertiary head & subsubsection environment \\ \hline
6. Abstract & abstract environment \\ \hline 
& \\ \hline
7. Body & regular text\\ \hline
8. Equation & equation environment\\ \hline
9.1 Figure & align left \\ \hline
9.2 Figure caption & caption environment \\ \hline
10.1 Table head & caption environment \\ \hline
10.2 Table text & tabular environment \\ \hline
11. References & bibliography + 
command \emph{small} to have a 1pt smaller font\\ \hline
12. Author Bio text & biography environment\\ \hline
13.1 List, bullet & itemize environment\\ \hline
13.2 List, numbered & enumerate environment \\ \hline
\end{tabular}
\end{center}
\end{table} 

%%%%%%%%%%%%%%%%%%%%%%%%%%%%%%%%%%
% Submitting Your Paper
%%%%%%%%%%%%%%%%%%%%%%%%%%%%%%%%%%

\section{Submitting Your Paper}
The submission of your paper has to be performed through the IS\&T
submission website. Authors receive a login for this site by e-mail.
Papers can be submitted in Postscript, Word, or WordPerfect format,
and will be converted to PDF by the submission server. Please
carefully review the generated PDF and verify that all the text,
equations, figures and tables are displayed correctly before approving
its submission.

\subsection{Margins in LaTeX}
Because of the differences between dvips conversion utilities, the
margins of your generated PDF document might vary. Please print
your document, and verify its margins. If they are incorrect, please
adapt the sizes of the margins in the file \emph{ist.sty}. Typically the
top margin should be decreased.

\begin{table}[!ht]
\caption{Document Specs: Table head}
\label{tab:specs}
\begin{center}       
\begin{tabular}{p{0.35\columnwidth}p{0.55\columnwidth}} 
Paper Size & US Letter \\
Left/right margin & .75 inch (1.90 cm) \\
Top/bottom margin & 1 inch (2.54 cm) \\
Columns & 2 at 3.38 inch (8.57 cm) wide. \\
 & Spacing between columns: 0.25 inch (0.635 cm)
\end{tabular}
\end{center}
\end{table} 

\begin{figure}[!hb]
%  \includegraphics[width=0.3\columnwidth]{logo.png}
  \caption{IS\&T logo.}
  \label{Figure:logo}
\end{figure}

Please contact IS\&T with any questions or requests for assistance in
helping prepare the paper. We look forward to having your paper
presented at the conference and published in the conference
proceedings.

\begin{figure}[!hb]
%  \includegraphics[width=0.3\columnwidth]{logo.png}
  \caption{IS\&T logo.}
  \label{Figure:logo}
\end{figure}

%%%%%%%%%%%%%%%%%%%%%%%%%%%%%%%%%%
% Reference Preparation
%%%%%%%%%%%%%%%%%%%%%%%%%%%%%%%%%%

\section{Reference Preparation}
Use the standard LaTeX \emph{cite} command for references in the
text. You can then use the standard bibliography command to generate
the list of references. Add the command \emph{small} before the
bibliography to give it the right font size.  Reference \cite{bib1}
style should be used for books, Reference \cite{bib2} style should be
used for Journals, and Reference \cite{bib3} style should be used for
Proceedings.

%\section{Acknowledgments} 
%add the acknowledgement section here

% To start a new column (but not a new page) and help balance the last-page
% column length use \vfill\pagebreak.

%%%%%%%%%%%%%%%%%%%%%%%%%%%%%%%%%%
% Bibliography
%%%%%%%%%%%%%%%%%%%%%%%%%%%%%%%%%%

\small
\begin{thebibliography}{9}
\bibitem{bib1}John Doe, Recent Progress in Digital Halftoning II,
  IS\&T, Springfield, VA, 1999, pg. 173.
\bibitem{bib2}John Doe, Digital Imaging, J. Imaging. Sci. and
  Technol., 42, 112 (1998).
\bibitem{bib3}John Doe, An Inexpensive Micro-Goniophotometry You Can
  Build, Proc. PICS, pg. 179. (1998).
\end{thebibliography}

%%%%%%%%%%%%%%%%%%%%%%%%%%%%%%%%%%
% Biography
%%%%%%%%%%%%%%%%%%%%%%%%%%%%%%%%%%

\begin{biography}
Please submit a brief biographical sketch of no more than 75 words. 
Include relevant professional and educational information as shown 
in the example below.

Jane Doe received her BS in physics from the University of Nevada (1977) 
and her PhD in applied physics from Columbia University (1983). Since 
then she has worked in the Research and Technology Division at Xerox 
in Webster, NY. Her work has focused on the development of toner adhesion 
and transport issues. She is on the Board of  IS\&T and a member of APS 
and SPIE.
\end{biography}

\end{document} 
